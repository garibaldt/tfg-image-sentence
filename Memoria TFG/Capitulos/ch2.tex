%%%
%%% BACHELOR'S THESIS TEMPLATE - ENGLISH
%%%
%%%  * the second chapter
%%%
%%%  AUTHORS:  Arnost Komarek (komarek@karlin.mff.cuni.cz), 2011
%%%            Michal Kulich (kulich@karlin.mff.cuni.cz), 2013
%%%
%%%  LAST UPDATED: 20130318
%%%
%%%  ===========================================================================

\chapter{Trabajo relacionado}

%%%%% ===============================================================================
Tanto la tarea de an�lisis de im�genes como la de generaci�n de sentencias en lenguaje natural
mediante \textit{deep learning} est� en desarrollo constante, y cada mes aparece un nuevo art�culo
o tesis sobre este tema.

La idea de utilizar redes neuronales recurrentes para construir modelos relacionados
con el procesamiento del lenguaje natural est� muy presente en nuestro trabajo. En \citet{kombrink2011recurrent}
se propone un modelo que utiliza RNNs con este prop�sito, y analiza su comportamiento en esta tarea.
Existen numerosos estudios sobre la generaci�n de frases, donde nos interesan especialmente aquellos que
las construyen en base a unas etiquetas [\citet{2014neural}], m�s cercano a la idea de combinar
an�lisis de im�genes y sus descripciones (los elementos que aparecen en ellas se usan para generar las frases).

An�logamente, se ha probado que las redes neuronales convolucionales dan buenos resultados
cuando estamos tratando con datos en forma de imagen, como en \cite{2012imagenet}.
Sin embargo, tambi�n se plantea el problema de conseguir buenos tiempos de entrenamiento
en estas redes con tantos par�metros que ajustar. 

Para trabajar con las t�cnicas de \textit{deep learning}, han aparecido numerosas APIs
que automatizan gran parte de los c�lculos necesarios en la red. Entre ellas destacamos
Theano (la que usaremos en este proyecto) y TensorFlow, que cuenta con el apoyo de Google.

En el trabajo de \cite{matchwords} se explora la conexi�n entre im�genes (o regiones de im�genes) y palabras
mediante diferentes modelos, a�n sin utilizar t�cnicas de \textit{deep learning}, presentes en las publicaciones
m�s actuales.

Los trabajos que combinan ambos enfoques para generar descripciones de im�genes son m�s recientes,
Estos son los que nos interesan, y que vamos a tomar como base para construir nuestro propio modelo.
El elemento com�n entre todos ellos es que se basan en redes neuronales convolucionales para el an�lisis de im�genes
y redes neuronales recurrentes para la generaci�n de frases [\citet{karpathy2015deep, vinyals2015show}].
En el trabajo de \citet{karpathy2015deep} se trabaja con una modificaci�n de las RNNs tradicionales
(introduciendo la bidireccionalidad en la red) y de las CNNs, para que ambas trabajen bien en conjunto.

Tambi�n es importante destacar la importancia de contar con buenos \textit{datsets} y metodolog�as
definidas para la evaluaci�n de parejas imagen-frase. En el trabajo de Hodosh et al. \cite{hodosh2013framing} se
recopilan anotaciones de 8000 im�genes (5 para cada una de las im�genes) que deben describir las entidades y los eventos.
Como en el resto de \textit{datasets} que vamos a estudiar en este trabajo, las anotaciones se han obtenido mediante trabajadores
humanos con plataformas de \textit{crowdsourcing}. Este \textit{dataset} se denomina Flickr8k. Mencionamos el trabajo relacionado
con otro de los \textit{datsets} que vamos a utilizar: MSCOCO \citet{mscoco}.
