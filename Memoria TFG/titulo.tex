%%%
%%% BACHELOR'S THESIS TEMPLATE - ENGLISH
%%%
%%%  * the title page and front matter
%%%
%%%  AUTHORS:  Martin Mares (mares@kam.mff.cuni.cz)
%%%            Arnost Komarek (komarek@karlin.mff.cuni.cz), 2011
%%%            Michal Kulich (kulich@karlin.mff.cuni.cz), 2013
%%%
%%%  LAST UPDATED: 20130318
%%%
%%%  ===========================================================================

\pagestyle{empty}
\begin{center}

{\large Universidad Complutense de Madrid}

\medskip
{\large Facultad de Inform�tica}

\vfill
{\bfseries\Large TRABAJO DE FIN DE GRADO}

\vfill
\centerline{\mbox{\includegraphics[width=60mm]{\FIGDIR/ucm_icono.eps}}}

\vfill
\vspace{5mm}

{\LARGE Daniel Gamo Alonso}

\vspace{15mm}

% Title in English according to the official assignment
{\LARGE\bfseries Generaci�n de descripciones de im�genes mediante \textit{Deep Learning}}

\vfill

Doble Grado en Ingenier�a Inform�tica y Matem�ticase
%%% department that confirmed the assignment of the thesis
%%% according to the Internal Structure of MFF UK in English
%%% see http://www.mff.cuni.cz/toUTF8.en/fakulta/struktura/sekcem.htm
% Department of Algebra
% Department of Mathematics Education
% Department of Mathematical Analysis
% Department of Numerical Mathematics
% Department of Probability and Mathematical Statistics
% Mathematical Institute of Charles University

\vfill

\begin{tabular}{rl}

\noalign{\vspace{2mm}}
Directores: & Alberto D�az Esteban, Gonzalo M�ndez\\

\end{tabular}

\vfill

% Fill the year
5 de julio de 2017

\end{center}


%%% At this place, the printed version includes a page containing the
%%% photocopy of the official signed "Bachelor thesis assignment".
%%% This should not be included in the electronic version.


%%% Acknowledgments
\newpage
\openright

\noindent
TODO - Agradecimientos

\newpage
%%% Czech and English abstracts

\vbox to 0.5\vsize{
\setlength\parindent{0mm}
\setlength\parskip{5mm}

\begin{center}
Resumen
\end{center}

El objetivo de este trabajo es construir un sistema basado en \textit{deep learning}
con el prop�sito de generar descripciones textuales a partir de las im�genes que se
suministren como entrada. Para ello se utilizar�n distintas t�cnicas relacionadas
con la visi�n artificial y el procesamiento del lenguaje natural como redes
neuronales y otras metodolog�as basadas en modelos
estad�sticos para, a partir de los elementos presentes en las
im�genes, predecir y generar descripciones coherentes con su contenido. Entre
las posibles aplicaciones podr�a estar el facilitar la accesibilidad a contenido
en forma de imagen a  usuarios con alg�n tipo de discapacidad visual.
\\
\\
\textbf{Palabras clave:} \textit{deep learning}, red neuronal, descripci�n de im�genes.


TODO: ENGLISH

\vss}

%%% A page containing the automatically generated contents of the
%%% bachelor thesis. For a mathematical thesis it is allowed to
%%% place the list of tables and abbreviations at the beginning
%%% of the thesis instead of at the end.
\newpage
\openright

\pagestyle{plain}
\setcounter{page}{1}

\tableofcontents
